\documentclass{article}
%% PLEASE DO NOT EDIT THIS FILE UNLESS INSTRUCTED TO DO SO!
\usepackage[utf8]{inputenc}
\usepackage{fullpage}
\usepackage{enumitem}
\usepackage{verbatim}
\usepackage{xcolor}
\usepackage{amsmath,amssymb,amsthm}
\usepackage{pgf,tikz}
\usepackage{mathrsfs}
\usetikzlibrary{arrows}

\newcommand{\todoG}[1]
{\textit{\textcolor{green!60!black}{#1}}}

\setlist[enumerate]{itemsep=3pt,topsep=3pt}
\setlist[enumerate,1]{label=(\alph*)}

\renewcommand\baselinestretch{1.5}

\newcommand{\WAtitle}[2]{\noindent\textbf{\Large Writing Assignment \#{#1} \hfill due #2}\\}
\newcommand{\WAauthors}[1]{\noindent {#1}\\}
\newcommand{\defi}[1]{\textbf{\textit{#1}}}

\theoremstyle{definition}
\newtheorem*{defn*}{Definition}
\newtheorem*{prop*}{Proposition}
\newtheorem*{example*}{Example}
\newtheorem*{thm*}{Theorem}
\newtheorem*{lemma*}{Lemma}

\DeclareMathOperator{\Row}{row}
\DeclareMathOperator{\Col}{col}
\DeclareMathOperator{\trace}{Tr}

\begin{document}

\WAtitle{NUMBER}{DATE} %% Put the assignment number in the first { } and the due date in the second { }
\WAauthors{AUTHORS} %% Put the authors here alphabetically by last name

\todoG{
    This is a solo assignment. Please delete the green stuff enclosed in the \LaTeX\ code 
    ``\textbackslash todoG\{ ... \}'' before turning it in.
    }

\todoG{
    The first problem is to write a paragraph that explains how to arrive at the quadratic formula
    by completing the square. You can look up how to do it--this isn't about the math, 
    it's about the writing about math.
    }

\todoG{
    Below is an example of a similar paragraph explaining how to arrive 
    at the slope-intercept form of a line from point-slope.
    }

    Suppose that $\ell$ is a line with a slope of $m$ that passes through the point $(x_0, y_0)$. 
    Traditionally, we write lines in ``slope-intercept'' form, 
    $y = mx + b$, where $m$ the ordered pair $(0,b)$ is the point on the line where it passes through the $y$-axis.  
    The point-slope formula for our line can be manipulated to produce the slope-intercept form using 
    the properties of real number addition and multiplication. 
    Indeed, $y-y_1  = m(x-x_1)$ implies $y = mx + (y_1 - mx_1)$, from which we see that $b = y_1 - mx_1$.

\todoG{Here's a sample start to your paragraph.}

Consider the quadratic equation $ax^2 + bx + c = 0$ where $a \not = 0$. Many people know of the existence of the quadratic formula, 
    \begin{equation}\label{eqn:QuadFormula} %%% Names this equation for later reference
        x = \frac{-b \pm \sqrt{b^2 - 4ac}}{2a},
    \end{equation}
for finding the solutions to this equation.

\todoG{
    From here, tell your reader that you're going to arrive at the quadratic formula 
    through a process called completing the square, show the big ideas 
    (without a lot of small arithmetic steps because your reader can, and should, 
    do a little scratchwork to follow your outline), 
    and probably this means a couple of different equations that show how
    $ax^2 + bx + c = 0$ implies Equation~\eqref{eqn:QuadFormula}.
    }

\todoG{
    Notice in the code on line 35, I gave you a short code to label an equation 
    (it's the \texttt{label} command after the \texttt{begin\{equation\}})
    and then to its number later at the end of line 46 
    (that's the \texttt{eqref} command, into which you plug in the name from the label.
    This is a really handy built-in system that \LaTeX\ offers so that you don't have to 
    manually update equation numbers and references to them when you're working on a long document!
    }

\clearpage

%%% WA 01 %%%

\WAtitle{NUMBER}{DATE} %% Put the assignment number in the first { } and the due date in the second { }
\WAauthors{AUTHORS} %% Put the authors here alphabetically by last name

\begin{prop*}[Problem CG 2025.] For all positive integers $n$,
\[
    \sum_{j=1}^n j^3 = \left(\sum_{j=1}^n\right)^2.
\]

In particular, \[2025 = 45^2 = (1 + 2 + 3 + 4 + 5 + 6 + 7 + 8 + 9)^2 = 1^3 + 2^3 + 3^3 + 4^3 + 5^3 + 6^3 + 7^3 + 8^3 + 9^3.\]
\end{prop*}

\begin{proof} 
    We proceed by induction. Let $n$ be a positive integer.

    % Base Case:
    When $n = 1$, %%% Finish the base case: $\sum_{j=1}^1 j^3 = 1^3 = 1 = 1^2 = \left(\sum_{j=1}^1 j\right)^2$.

    % Induction Hypothesis: assume the formula is true for $k \geq 1$, then show it holds for $k + 1$.
    Let$k \geq 1$ and assume
    %%% Fill in the induction hypothesis

    %%% Complete the induction step;
    %%% it might be helpful to use a result we proved in class to simplify
    %%% the sum inside the square:
    %%% $\sum_{j=1}^k j = \frac{k(k+1)}{2}$.

    Therefore, by the principle of mathematical induction, %%% Fill out the conclusion
\end{proof}

\begin{prop*}[Chapter 1, \#39]. Let $m$ and $n$ be nonzero integers, and assume $m$ and $k$ are relatively prime integers. If $k$ is an integer for which $m$ divides $nk$, then $m$ divides $k$.
\end{prop*}

\begin{proof}

%%% Don't forget to make sure you're satisfying the hypotheses of any theorems you use (like the FTa or the GCD theorem).
%%% You might need to take care of some special cases depending on what route you choose!

\end{proof}

\todoG{The next problems involve \textbf{binomial coefficients}, which are defined in our book right around where these problems are stated. They're the numbers that show up in Pascal's Triangle (still attributed to Blaise Pascal for some reason even though it was known millenia before by at least 3 distinct civilizations\dots}

\begin{lemma*}[Chapter 1, \#41] For all positive integers $n$ and $r$ satisfying $0 < r \leq n$, the binomial coefficients satisfy the equality \[
\binom{n+1}{r} = \binom{n}{r-1} + \binom{n}{r}\]
(where $\binom{n}{r} = \frac{n!}{(n-r)!r!}$ and $0! = 1$ by definition).
\end{lemma*}

\todoG{You can use the above lemma without proving it. You can also use the result that binomial coefficients are integers without proof.}

\begin{example*}[Chapter 1, \#43 and \#44] The Binomial Theorem helps us expand $(x+1)^5$.  Indeed, 
    \begin{align*} 
        (x+1)^5 &= \sum_{k=0}^5 \binom{5}{k} x^{5-k}1^k \\ 
                %% ^^^ fixed 9/8/2021 (bad Gibbons!)
                &= \binom{5}{0} x^5 + \binom{5}{1}x^4 + \todoG{...} + \binom{5}{5}\\
                &= x^5 + 5x^4 + \todoG{...} + 1.
    \end{align*}
   
    \todoG{Tell your reader what you notice about the middle coefficients; use \#43 for inspiration here. Hint: 5 is prime.}
\end{example*}

\clearpage

%%% WA 02 %%%

\WAtitle{NUMBER}{DATE}
\WAauthors{AUTHORS}

\begin{lemma*}[Ch. 1, \#39] 
    For all nonzero integers $m$ and $n$, if $\gcd(m,n) = 1$ and $n$ divides $mk$ for some integer $k$, then $n$ divides $k$.
\end{lemma*}

\todoG{
    You proved this on the previous assignment, so you don't need to prove it again. 
    I just put it here for your reference because I think you'll need it for \#56
    }

\begin{prop*}[Ch. 3, \#53] ...
\end{prop*}

\begin{prop*}[Ch. 3, \#56] ...
\end{prop*}

\end{document}

There are useful pieces of code in the file WA-0.tex if you get stuck!